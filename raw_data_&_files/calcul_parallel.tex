% Options for packages loaded elsewhere
\PassOptionsToPackage{unicode}{hyperref}
\PassOptionsToPackage{hyphens}{url}
%
\documentclass[
]{article}
\usepackage{amsmath,amssymb}
\usepackage{iftex}
\ifPDFTeX
  \usepackage[T1]{fontenc}
  \usepackage[utf8]{inputenc}
  \usepackage{textcomp} % provide euro and other symbols
\else % if luatex or xetex
  \usepackage{unicode-math} % this also loads fontspec
  \defaultfontfeatures{Scale=MatchLowercase}
  \defaultfontfeatures[\rmfamily]{Ligatures=TeX,Scale=1}
\fi
\usepackage{lmodern}
\ifPDFTeX\else
  % xetex/luatex font selection
\fi
% Use upquote if available, for straight quotes in verbatim environments
\IfFileExists{upquote.sty}{\usepackage{upquote}}{}
\IfFileExists{microtype.sty}{% use microtype if available
  \usepackage[]{microtype}
  \UseMicrotypeSet[protrusion]{basicmath} % disable protrusion for tt fonts
}{}
\makeatletter
\@ifundefined{KOMAClassName}{% if non-KOMA class
  \IfFileExists{parskip.sty}{%
    \usepackage{parskip}
  }{% else
    \setlength{\parindent}{0pt}
    \setlength{\parskip}{6pt plus 2pt minus 1pt}}
}{% if KOMA class
  \KOMAoptions{parskip=half}}
\makeatother
\usepackage{xcolor}
\usepackage[margin=1in]{geometry}
\usepackage{color}
\usepackage{fancyvrb}
\newcommand{\VerbBar}{|}
\newcommand{\VERB}{\Verb[commandchars=\\\{\}]}
\DefineVerbatimEnvironment{Highlighting}{Verbatim}{commandchars=\\\{\}}
% Add ',fontsize=\small' for more characters per line
\usepackage{framed}
\definecolor{shadecolor}{RGB}{248,248,248}
\newenvironment{Shaded}{\begin{snugshade}}{\end{snugshade}}
\newcommand{\AlertTok}[1]{\textcolor[rgb]{0.94,0.16,0.16}{#1}}
\newcommand{\AnnotationTok}[1]{\textcolor[rgb]{0.56,0.35,0.01}{\textbf{\textit{#1}}}}
\newcommand{\AttributeTok}[1]{\textcolor[rgb]{0.13,0.29,0.53}{#1}}
\newcommand{\BaseNTok}[1]{\textcolor[rgb]{0.00,0.00,0.81}{#1}}
\newcommand{\BuiltInTok}[1]{#1}
\newcommand{\CharTok}[1]{\textcolor[rgb]{0.31,0.60,0.02}{#1}}
\newcommand{\CommentTok}[1]{\textcolor[rgb]{0.56,0.35,0.01}{\textit{#1}}}
\newcommand{\CommentVarTok}[1]{\textcolor[rgb]{0.56,0.35,0.01}{\textbf{\textit{#1}}}}
\newcommand{\ConstantTok}[1]{\textcolor[rgb]{0.56,0.35,0.01}{#1}}
\newcommand{\ControlFlowTok}[1]{\textcolor[rgb]{0.13,0.29,0.53}{\textbf{#1}}}
\newcommand{\DataTypeTok}[1]{\textcolor[rgb]{0.13,0.29,0.53}{#1}}
\newcommand{\DecValTok}[1]{\textcolor[rgb]{0.00,0.00,0.81}{#1}}
\newcommand{\DocumentationTok}[1]{\textcolor[rgb]{0.56,0.35,0.01}{\textbf{\textit{#1}}}}
\newcommand{\ErrorTok}[1]{\textcolor[rgb]{0.64,0.00,0.00}{\textbf{#1}}}
\newcommand{\ExtensionTok}[1]{#1}
\newcommand{\FloatTok}[1]{\textcolor[rgb]{0.00,0.00,0.81}{#1}}
\newcommand{\FunctionTok}[1]{\textcolor[rgb]{0.13,0.29,0.53}{\textbf{#1}}}
\newcommand{\ImportTok}[1]{#1}
\newcommand{\InformationTok}[1]{\textcolor[rgb]{0.56,0.35,0.01}{\textbf{\textit{#1}}}}
\newcommand{\KeywordTok}[1]{\textcolor[rgb]{0.13,0.29,0.53}{\textbf{#1}}}
\newcommand{\NormalTok}[1]{#1}
\newcommand{\OperatorTok}[1]{\textcolor[rgb]{0.81,0.36,0.00}{\textbf{#1}}}
\newcommand{\OtherTok}[1]{\textcolor[rgb]{0.56,0.35,0.01}{#1}}
\newcommand{\PreprocessorTok}[1]{\textcolor[rgb]{0.56,0.35,0.01}{\textit{#1}}}
\newcommand{\RegionMarkerTok}[1]{#1}
\newcommand{\SpecialCharTok}[1]{\textcolor[rgb]{0.81,0.36,0.00}{\textbf{#1}}}
\newcommand{\SpecialStringTok}[1]{\textcolor[rgb]{0.31,0.60,0.02}{#1}}
\newcommand{\StringTok}[1]{\textcolor[rgb]{0.31,0.60,0.02}{#1}}
\newcommand{\VariableTok}[1]{\textcolor[rgb]{0.00,0.00,0.00}{#1}}
\newcommand{\VerbatimStringTok}[1]{\textcolor[rgb]{0.31,0.60,0.02}{#1}}
\newcommand{\WarningTok}[1]{\textcolor[rgb]{0.56,0.35,0.01}{\textbf{\textit{#1}}}}
\usepackage{graphicx}
\makeatletter
\def\maxwidth{\ifdim\Gin@nat@width>\linewidth\linewidth\else\Gin@nat@width\fi}
\def\maxheight{\ifdim\Gin@nat@height>\textheight\textheight\else\Gin@nat@height\fi}
\makeatother
% Scale images if necessary, so that they will not overflow the page
% margins by default, and it is still possible to overwrite the defaults
% using explicit options in \includegraphics[width, height, ...]{}
\setkeys{Gin}{width=\maxwidth,height=\maxheight,keepaspectratio}
% Set default figure placement to htbp
\makeatletter
\def\fps@figure{htbp}
\makeatother
\setlength{\emergencystretch}{3em} % prevent overfull lines
\providecommand{\tightlist}{%
  \setlength{\itemsep}{0pt}\setlength{\parskip}{0pt}}
\setcounter{secnumdepth}{5}
\ifLuaTeX
  \usepackage{selnolig}  % disable illegal ligatures
\fi
\IfFileExists{bookmark.sty}{\usepackage{bookmark}}{\usepackage{hyperref}}
\IfFileExists{xurl.sty}{\usepackage{xurl}}{} % add URL line breaks if available
\urlstyle{same}
\hypersetup{
  pdftitle={TP Calcul Parallel},
  hidelinks,
  pdfcreator={LaTeX via pandoc}}

\title{TP Calcul Parallel - Code Rmarkdown}
\author{}
\date{\vspace{-2.5em}}
\usepackage[french]{babel}
\begin{document}
\maketitle
\hrulefill
{
\setcounter{tocdepth}{2}
\tableofcontents
}
\hrulefill

\newpage

\hypertarget{objectifs-et-importance-de-la-progammation-paralluxe8le}{%
\section{Objectifs et Importance de la progammation
parallèle}\label{objectifs-et-importance-de-la-progammation-paralluxe8le}}

De manière générale, l'exécution d'une opération sur un ordinateur se
fait suivant le principe du calcul séquentiel qui consiste en
l'exécution de l'opération à travers des étapes successives, où chaque
étape ne se déclenche que lorsque l'étape précédente est terminée, y
compris lorsque les deux étapes sont indépendantes a priori sur une
seule ressource. L'implémentation d'un programme sur R obéit par défaut
au principe du calcul séquentielle. Ceci étant, à mesure que les
opérations effectuées portent sur des jeux de données relativement
grands, ce principe de calcul révèle un certain nombre de limites:

\begin{itemize}
\item
  Il est trop coûteux en temps de calcul, en mémoire;
\item
  Les volumes de données à traiter sont trop importants, trop longs à
  écrire;
\item
  Les performances sont moins bonnes que sur des machines plus
  vieilles,\ldots{}
\end{itemize}

Lorsque l'exécution d'un programme sur R s'avère lent aux besoins de ses
utilisateurs, son temps d'exécution doit être optimisé. Il existe
plusieurs stratégies pour arriver à cette optimisation: il est
recommandé utiliser des fonctions déjà optimisées et disponibles
publiquement ; exploiter les calculs vectoriels et matriciels, qui sont
plus rapides que des boucles en R ; éviter les allocations mémoire
inutiles, notamment les objets de taille croissante et les modifications
répétées d'éléments dans un data frame. Malgré ces méthodes
d'optimisation, la programmation peut toujours être aussi lente. Une
solution appropriée sur R pour optimiser le temps d'exécution est alors
le calcul en parallèle.

Ainsi, l'objectif du calcul en parallèle est d'effectuer plus rapidement
un calcul informatique en exploitant simultanément plusieurs unités de
calcul.

\hypertarget{principe-guxe9nuxe9ral}{%
\section{Principe général}\label{principe-guxe9nuxe9ral}}

\begin{enumerate}
\def\labelenumi{\arabic{enumi}.}
\item
  Briser un calcul informatique en blocs de calcul indépendants;
\item
  Exécuter simultanément (en parallèle) les blocs de calcul sur
  plusieurs unités de calcul;
\item
  Rassembler les résultats et les retourner.
\end{enumerate}

\textbf{Paralléliser} un problèmeconsiste à décomposer ce problème en
plusieurs sous problèmes à résoudre simultanément à travers différentes
ressources, pour ressortir la solution du problème initial, dans un
délai optimal. Ainsi, le principe du \textbf{calcul parallèle} est
d'effectuer simultanément une même tâche ou exécuter un même programme
de manière parallèle. Cela est aussi possible à travers différentes
machines connectées par un réseau où chacun d'eux reçoit une tache à
exécuter. Sur R, L'utilité du calcul en parallèle réside dans le faite
qu'il permet d'effectuer plus rapidement et de manière asynchrone
l'exécution de programme sur des bases de données volumineuses en
exploitant simultanément plusieurs unités de calcul d'un ordinateur
appelées cœurs.

\begin{figure}
\centering
\includegraphics{images/map.PNG}
\caption{Modèle de programmation MapReduce}
\end{figure}

\hypertarget{notions-de-base}{%
\section{Notions de base}\label{notions-de-base}}

\begin{enumerate}
\def\labelenumi{\arabic{enumi}.}
\item
  \textbf{Processeur ou CPU:} Son rôle est de lire et d'exécuter les
  instructionsprovenant d'un programme.

  \begin{itemize}
  \item
    Les processeurs sont de nos jours la plupart du temps
    \textbf{divisés en plus d'une unité de calcul}, nommée coeur (en
    anglais core). Il s'agit alors de \textbf{processeurs multi-coeurs}.
    Ce type de matériel permet de faire du calcul en parallèle sur une
    seule machine, en exploitant plus d'un coeur de la machine.
  \item
    \textbf{Threads:} Les coeurs exécutent ce que l'on appelle des
    \textbf{fils d'exécution} (en anglais \textbf{threads}). Un fil
    d'exécution est une petite séquence d'instructions en langage
    machine.
  \item
    \textbf{Processus monothread}: Lorsque les fils d'exécution sont
    exécutés séquentiellement par un coeur soit un après l'autre.
  \item
    \textbf{Processus multithreads:} Il existe cependant une technologie
    permettant à un seul coeur physique d'exécuter plus d'un fil
    d'exécution simultanément. On dit alors que le coeur physique est
    séparé en \textbf{coeurs logiques}. On parle alors d'un coeur
    multithread.
  \end{itemize}
\item Hadoop: Hadoop est un framework open source largement utilisé pour le traitement et le stockage distribués de données volumineuses (big data). Il fournit une plate-forme pour le traitement parallèle de grandes quantités de données en les répartissant sur un cluster de serveurs.
\end{enumerate}

\hypertarget{etapes-du-calcul-paralluxe8le}{%
\section{Etapes du calcul
parallèle}\label{etapes-du-calcul-paralluxe8le}}

\begin{enumerate}
\def\labelenumi{\arabic{enumi}.}
\item
  Démarrer m processus ``travailleurs'' (i.e.cœurs de calcul) et les
  initialiser;
\item
  Envoyer les fonctions et données nécessaires pour chaque tache aux
  travailleurs;
\item
  Séparer les taches en m opérations d'envergure similaire et les
  envoyer aux travailleurs;
\item
  Attendre que tous les travailleurs aient terminer leurs calculs et
  obtenir leurs résultats;
\item
  Rassembler les résultats des différents travailleurs;
\item
  Arrêter les processus travailleurs
\end{enumerate}

Le package parallel permet de démarrer et d'arrêter un ``cluster'' de
plusieurs processus travailleur (étape 1). En plus du package
\texttt{parallel}, on va donc utiliser le package \texttt{doParallel}
qui permet de gérer les processus travailleurs et la communication
(étapes 1) et l'articulation avec le package \texttt{foreach}qui permet
lui de gérer le dialogue avec les travailleurs (envois, réception et
rassemblement des résultats - étapes 2, 3, 4 et 5).

\hypertarget{programmation-paralluxe8le-avec-r}{%
\section{Programmation parallèle avec
R}\label{programmation-paralluxe8le-avec-r}}

\begin{itemize}
\item
  Package \textbf{Parallel:} inclus dans la distribution de base de R:
  Il se base sur l'utilisation de fonctions de la famille des apply.
\item
  Package doParallel et Foreach:
\item
  Le package rmr2 (MapReduce)
\item
  Etc.
\end{itemize}

\hypertarget{quelques-fonctions-importantes}{%
\section{Quelques fonctions
importantes}\label{quelques-fonctions-importantes}}

\begin{itemize}
\item
  Du package Parallel

  \begin{itemize}
  \item
    \textbf{Detectcores()}: permet de détecter le nombre cœurs de la
    machine.
  \item
    \textbf{Makecluster() :}
  \item
    \textbf{Stopcluster():} est utilisé pour arrêter et libérer les
    différents workers.
  \item
    La famille des fonctions \textbf{Applay,} adaptées au calcul
    parallèle sous R permet d'exécuter simultanément les opérations sur
    les différents blocs.

    \begin{itemize}
    \item
      parApplay() permet d'effectuer des calculs en parallèle sur une
      matrice ou un tableau

      en utilisant un cluster de travailleurs
    \item
      parLapply() permet d'appliquer une fonction à chaque élément d'une
      liste en utilisant un cluster de travailleurs pour exécuter les
      calculs en parallèle.
    \item
      parSapplay() permet d'appliquer une fonction de manière parallèle
      à des éléments d'une liste.Elle prend en argument le jeu de donné
      et la fonction et retourne un vecteur ou une matrice.
    \end{itemize}
  \item
    clusterEvalQ(): Elle est utilisée pour évaluer une expression sur
    tous les nœuds d'un cluster parallèle. Elle est utile lorsque vous
    avez besoin d'exécuter une expression ou de charger des
    bibliothèques spécifiques sur chaque nœud du cluster avant
    d'exécuter des tâches parallèles.
  \end{itemize}
\item
  Du package Doparallel et Foreach

  \begin{itemize}
  \item
    Foreach: Il constitue une alternative aux fonctions applay utilisé
    dans le package parallèle. Il fournit une approche simplifiée pour
    effectuer des boucles parallèles en R, en permettant d'exploiter
    efficacement les ressources de calcul disponibles sur un système. Il
    s'appuie généralement sur d'autres packages parallèles, tels que
    ``doParallel'' ou ``doSNOW'', pour exécuter les boucles en
    parallèle.
  \item
    \textbf{\texttt{\%dopar\%}} est un opérateur spécifique du package
    ``foreach'' en R, qui permet d'effectuer des itérations parallèles
    sur des objets itérables tels que des vecteurs, des listes ou des
    data frames.
  \item
    registerDoParallel() fait le même que Makecluster
  \end{itemize}
\item
  Package rmr2 (Disponible seulement pour les versions antérieures de R)

  \begin{itemize}
  \item
    mapreduce(): La fonction \textbf{\texttt{mapreduce()}} est une
    fonction clé du package R ``rmr2'' (ou ``RHadoop'') qui fournit une
    interface pour exécuter des calculs distribués sur des systèmes de
    fichiers distribués tels que Hadoop.
  \item
    keyval(): elle est utilisée pour définir des paires clé-valeur qui
    serviront de données d'entrée pour les opérations de MapReduce.
    Cette fonction prend deux arguments : une clé et une valeur, et
    retourne une structure de données représentant une paire clé-valeur.
  \end{itemize}
\end{itemize}

\hypertarget{pruxe9liminaire}{%
\section{Préliminaire}\label{pruxe9liminaire}}
\subsection{garbage collector}
\begin{Shaded}
\begin{Highlighting}[]
\CommentTok{\#vider la mémoire}
\FunctionTok{rm}\NormalTok{(}\AttributeTok{list=}\FunctionTok{ls}\NormalTok{())}
\end{Highlighting}
\end{Shaded}

\begin{Shaded}
\begin{Highlighting}[]
\CommentTok{\#lancer le garbage collector}
\FunctionTok{gc}\NormalTok{()}
\end{Highlighting}
\end{Shaded}

\begin{verbatim}
##          used (Mb) gc trigger (Mb) max used (Mb)
## Ncells 520845 27.9    1166272 62.3   643711 34.4
## Vcells 911292  7.0    8388608 64.0  1648775 12.6
\end{verbatim}

Le garbage collector permet de gérer automatiquement la mémoire allouée
aux objets.

Lorsqu'un programme s'exécute, il alloue de la mémoire pour créer des
objets et stocker des données. Cependant, il arrive souvent que certains
objets ne soient plus utilisés par le programme, ce qui crée des
``déchets'' ou des ``objets morts'' en mémoire. Si ces objets morts ne
sont pas libérés, ils peuvent occuper de l'espace précieux en mémoire et
entraîner des problèmes tels que des fuites de mémoire.

\hypertarget{packages}{%
\subsection{Packages}\label{packages}}

\begin{Shaded}
\begin{Highlighting}[]
\CommentTok{\#information sur les versions}
\FunctionTok{sessionInfo}\NormalTok{()}
\end{Highlighting}
\end{Shaded}

\begin{verbatim}
## R version 4.1.2 (2021-11-01)
## Platform: x86_64-w64-mingw32/x64 (64-bit)
## Running under: Windows 10 x64 (build 19045)
## 
## Matrix products: default
## 
## locale:
## [1] LC_COLLATE=French_France.1252  LC_CTYPE=French_France.1252   
## [3] LC_MONETARY=French_France.1252 LC_NUMERIC=C                  
## [5] LC_TIME=French_France.1252    
## 
## attached base packages:
## [1] parallel  stats     graphics  grDevices utils     datasets  methods  
## [8] base     
## 
## other attached packages:
## [1] snow_0.4-4        doParallel_1.0.17 iterators_1.0.14  foreach_1.5.2    
## [5] tictoc_1.2       
## 
## loaded via a namespace (and not attached):
##  [1] codetools_0.2-18 digest_0.6.29    lifecycle_1.0.3  magrittr_2.0.3  
##  [5] evaluate_0.21    rlang_1.1.0      stringi_1.7.12   cli_3.1.1       
##  [9] rstudioapi_0.14  vctrs_0.6.1      rmarkdown_2.22   tools_4.1.2     
## [13] stringr_1.5.0    glue_1.6.2       xfun_0.39        yaml_2.2.2      
## [17] fastmap_1.1.0    compiler_4.1.2   htmltools_0.5.2  knitr_1.37
\end{verbatim}

\begin{Shaded}
\begin{Highlighting}[]
\CommentTok{\#help(package=\textquotesingle{}parallel\textquotesingle{})}
\end{Highlighting}
\end{Shaded}

\hypertarget{duxe9finition-de-la-fonction-pour-calculer-le-min}{%
\section{Définition de la fonction pour calculer le
Min}\label{duxe9finition-de-la-fonction-pour-calculer-le-min}}

\begin{Shaded}
\begin{Highlighting}[]
\NormalTok{mon\_min }\OtherTok{\textless{}{-}} \ControlFlowTok{function}\NormalTok{(v) \{}
  \CommentTok{\#copie locale}
\NormalTok{  temp }\OtherTok{\textless{}{-}}\NormalTok{ v}
  \CommentTok{\#longueur du vecteur}
\NormalTok{  n }\OtherTok{\textless{}{-}} \FunctionTok{length}\NormalTok{(temp)}
  \CommentTok{\#tri par selection si (n \textgreater{} 1)}
  \ControlFlowTok{if}\NormalTok{ (n }\SpecialCharTok{\textgreater{}} \DecValTok{1}\NormalTok{) \{}
    \CommentTok{\#recherche des minimums successifs}
    \ControlFlowTok{for}\NormalTok{ (i }\ControlFlowTok{in} \DecValTok{1}\SpecialCharTok{:}\NormalTok{(n }\SpecialCharTok{{-}} \DecValTok{1}\NormalTok{)) \{}
\NormalTok{      i\_mini }\OtherTok{\textless{}{-}}\NormalTok{ i}
      \ControlFlowTok{for}\NormalTok{ (j }\ControlFlowTok{in}\NormalTok{ (i }\SpecialCharTok{+} \DecValTok{1}\NormalTok{)}\SpecialCharTok{:}\NormalTok{n) \{}
        \ControlFlowTok{if}\NormalTok{ (temp[j] }\SpecialCharTok{\textless{}}\NormalTok{ temp[i\_mini]) \{}
\NormalTok{          i\_mini }\OtherTok{\textless{}{-}}\NormalTok{ j}
\NormalTok{        \}}
\NormalTok{      \}}
      \CommentTok{\#Echanger}
      \ControlFlowTok{if}\NormalTok{ (i\_mini }\SpecialCharTok{!=}\NormalTok{ i) \{}
\NormalTok{        tempo }\OtherTok{\textless{}{-}}\NormalTok{ temp[i]}
\NormalTok{        temp[i] }\OtherTok{\textless{}{-}}\NormalTok{ temp[i\_mini]}
\NormalTok{        temp[i\_mini] }\OtherTok{\textless{}{-}}\NormalTok{ tempo}
\NormalTok{      \}}
\NormalTok{    \}}
\NormalTok{  \}}
  \CommentTok{\#la plus petite valeur est le min.}
  \FunctionTok{return}\NormalTok{(temp[}\DecValTok{1}\NormalTok{])}
\NormalTok{\}}
\end{Highlighting}
\end{Shaded}

\hypertarget{application-de-la-programmation-paralluxe8le-pour-duxe9terminer-le-min}{%
\section{Application de la programmation parallèle pour déterminer le
Min}\label{application-de-la-programmation-paralluxe8le-pour-duxe9terminer-le-min}}

\begin{Shaded}
\begin{Highlighting}[]
\CommentTok{\# Génération d\textquotesingle{}un vecteur de données}
\NormalTok{n }\OtherTok{\textless{}{-}} \DecValTok{10}
\NormalTok{a }\OtherTok{\textless{}{-}} \FunctionTok{runif}\NormalTok{(n)}
\NormalTok{a}
\end{Highlighting}
\end{Shaded}

\begin{verbatim}
##  [1] 0.1268973 0.2190160 0.4429761 0.1406081 0.5884293 0.9038375 0.4424940
##  [8] 0.9737582 0.7543901 0.4391870
\end{verbatim}

\hypertarget{calcul-direct-sans-paralleliser}{%
\subsection{Calcul direct (sans
paralleliser)}\label{calcul-direct-sans-paralleliser}}

\begin{Shaded}
\begin{Highlighting}[]
\CommentTok{\#appel de la fonction sur la totalité du vecteur}
\FunctionTok{tic}\NormalTok{()}
\FunctionTok{print}\NormalTok{(}\FunctionTok{paste}\NormalTok{(}\StringTok{\textquotesingle{}Min direct =\textquotesingle{}}\NormalTok{,}\FunctionTok{mon\_min}\NormalTok{(a)))}
\end{Highlighting}
\end{Shaded}

\begin{verbatim}
## [1] "Min direct = 0.126897253561765"
\end{verbatim}

\begin{Shaded}
\begin{Highlighting}[]
\FunctionTok{print}\NormalTok{(}\StringTok{\textquotesingle{}\textgreater{}\textgreater{} Temps de calcul {-} fonction mon\_min direct\textquotesingle{}}\NormalTok{)}
\end{Highlighting}
\end{Shaded}

\begin{verbatim}
## [1] ">> Temps de calcul - fonction mon_min direct"
\end{verbatim}

\begin{Shaded}
\begin{Highlighting}[]
\FunctionTok{toc}\NormalTok{()}
\end{Highlighting}
\end{Shaded}

\begin{verbatim}
## 0.2 sec elapsed
\end{verbatim}

\hypertarget{calcul-en-utilisant-la-programmation-paralluxe8le}{%
\subsection{Calcul en utilisant la programmation
parallèle}\label{calcul-en-utilisant-la-programmation-paralluxe8le}}

\begin{Shaded}
\begin{Highlighting}[]
\CommentTok{\#affichage nombre de coeurs dispo}
\FunctionTok{print}\NormalTok{(parallel}\SpecialCharTok{::}\FunctionTok{detectCores}\NormalTok{())}
\end{Highlighting}
\end{Shaded}

\begin{verbatim}
## [1] 4
\end{verbatim}

\begin{Shaded}
\begin{Highlighting}[]
\CommentTok{\#nombre de blocs des donnees = nombre de cores}
\NormalTok{k }\OtherTok{\textless{}{-}} \DecValTok{4}
\CommentTok{\#partition en blocs des donn?es}
\NormalTok{blocs }\OtherTok{\textless{}{-}} \FunctionTok{split}\NormalTok{(a,}\DecValTok{1}\SpecialCharTok{+}\NormalTok{(}\DecValTok{1}\SpecialCharTok{:}\NormalTok{n)}\SpecialCharTok{\%\%}\NormalTok{k)}
\FunctionTok{print}\NormalTok{(blocs)}
\end{Highlighting}
\end{Shaded}

\begin{verbatim}
## $`1`
## [1] 0.1406081 0.9737582
## 
## $`2`
## [1] 0.1268973 0.5884293 0.7543901
## 
## $`3`
## [1] 0.2190160 0.9038375 0.4391870
## 
## $`4`
## [1] 0.4429761 0.4424940
\end{verbatim}

\begin{Shaded}
\begin{Highlighting}[]
\CommentTok{\#appel de la fonction sur la totalité du vecteur}
\FunctionTok{tic}\NormalTok{()}
\FunctionTok{print}\NormalTok{(}\FunctionTok{paste}\NormalTok{(}\StringTok{\textquotesingle{}Min direct =\textquotesingle{}}\NormalTok{,}\FunctionTok{mon\_min}\NormalTok{(a)))}
\end{Highlighting}
\end{Shaded}

\begin{verbatim}
## [1] "Min direct = 0.126897253561765"
\end{verbatim}

\begin{Shaded}
\begin{Highlighting}[]
\FunctionTok{print}\NormalTok{(}\StringTok{\textquotesingle{}\textgreater{}\textgreater{} Temps de calcul {-} fonction mon\_min direct\textquotesingle{}}\NormalTok{)}
\end{Highlighting}
\end{Shaded}

\begin{verbatim}
## [1] ">> Temps de calcul - fonction mon_min direct"
\end{verbatim}

\begin{Shaded}
\begin{Highlighting}[]
\FunctionTok{toc}\NormalTok{()}
\end{Highlighting}
\end{Shaded}

\begin{verbatim}
## 0.02 sec elapsed
\end{verbatim}

\begin{Shaded}
\begin{Highlighting}[]
\CommentTok{\#pour mesurer le processus global de **parallel**}
\FunctionTok{tic}\NormalTok{()}
\CommentTok{\#Demarrage des moteurs (workers)}
\NormalTok{clust }\OtherTok{\textless{}{-}}\NormalTok{ parallel}\SpecialCharTok{::}\FunctionTok{makeCluster}\NormalTok{(}\DecValTok{4}\NormalTok{)}
\CommentTok{\#lancement des min en parallele}
\NormalTok{res }\OtherTok{\textless{}{-}}\NormalTok{ parallel}\SpecialCharTok{::}\FunctionTok{parSapply}\NormalTok{(clust,blocs,}\AttributeTok{FUN =}\NormalTok{ mon\_min)}
\CommentTok{\#résultats intermédiaires}
\FunctionTok{print}\NormalTok{(res)}
\end{Highlighting}
\end{Shaded}

\begin{verbatim}
##         1         2         3         4 
## 0.1406081 0.1268973 0.2190160 0.4424940
\end{verbatim}

\begin{Shaded}
\begin{Highlighting}[]
\CommentTok{\#fonction de consolidation}
\FunctionTok{print}\NormalTok{(}\FunctionTok{paste}\NormalTok{(}\StringTok{\textquotesingle{}Min parallel =\textquotesingle{}}\NormalTok{,}\FunctionTok{mon\_min}\NormalTok{(res)))}
\end{Highlighting}
\end{Shaded}

\begin{verbatim}
## [1] "Min parallel = 0.126897253561765"
\end{verbatim}

\begin{Shaded}
\begin{Highlighting}[]
\CommentTok{\#Eteindre les moteurs}
\NormalTok{parallel}\SpecialCharTok{::}\FunctionTok{stopCluster}\NormalTok{(clust)}
\CommentTok{\#affichage temps de calcul}
\FunctionTok{print}\NormalTok{(}\StringTok{\textquotesingle{}\textgreater{}\textgreater{} Temps de calcul total avec parSapply min par bloc\textquotesingle{}}\NormalTok{)}
\end{Highlighting}
\end{Shaded}

\begin{verbatim}
## [1] ">> Temps de calcul total avec parSapply min par bloc"
\end{verbatim}

\begin{Shaded}
\begin{Highlighting}[]
\CommentTok{\#temps de calcul}
\FunctionTok{toc}\NormalTok{()}
\end{Highlighting}
\end{Shaded}

\begin{verbatim}
## 3.31 sec elapsed
\end{verbatim}

\hypertarget{calcul-de-la-moyenne}{%
\subsubsection{Calcul de la moyenne}\label{calcul-de-la-moyenne}}

\begin{Shaded}
\begin{Highlighting}[]
\CommentTok{\#appel de la fonction sur la totalité du vecteur}
\FunctionTok{tic}\NormalTok{()}
\FunctionTok{print}\NormalTok{(}\FunctionTok{paste}\NormalTok{(}\StringTok{\textquotesingle{}Moyenne direct =\textquotesingle{}}\NormalTok{,}\FunctionTok{mean}\NormalTok{(a)))}
\end{Highlighting}
\end{Shaded}

\begin{verbatim}
## [1] "Moyenne direct = 0.503159347549081"
\end{verbatim}

\begin{Shaded}
\begin{Highlighting}[]
\FunctionTok{print}\NormalTok{(}\StringTok{\textquotesingle{}\textgreater{}\textgreater{} Temps de calcul {-} fonction moyenne direct\textquotesingle{}}\NormalTok{)}
\end{Highlighting}
\end{Shaded}

\begin{verbatim}
## [1] ">> Temps de calcul - fonction moyenne direct"
\end{verbatim}

\begin{Shaded}
\begin{Highlighting}[]
\FunctionTok{toc}\NormalTok{()}
\end{Highlighting}
\end{Shaded}

\begin{verbatim}
## 0.02 sec elapsed
\end{verbatim}

\begin{Shaded}
\begin{Highlighting}[]
\CommentTok{\#pour mesurer le processus global de **parallel**}
\FunctionTok{tic}\NormalTok{()}
\CommentTok{\#Demarrage des moteurs (workers)}
\NormalTok{clust }\OtherTok{\textless{}{-}}\NormalTok{ parallel}\SpecialCharTok{::}\FunctionTok{makeCluster}\NormalTok{(}\DecValTok{4}\NormalTok{)}
\CommentTok{\#lancement des min en parallele}
\NormalTok{res }\OtherTok{\textless{}{-}}\NormalTok{ parallel}\SpecialCharTok{::}\FunctionTok{parSapply}\NormalTok{(clust,blocs,}\AttributeTok{FUN =}\NormalTok{ mean)}
\NormalTok{poids}\OtherTok{\textless{}{-}}\NormalTok{parallel}\SpecialCharTok{::}\FunctionTok{parSapply}\NormalTok{(clust,blocs,}\AttributeTok{FUN =}\NormalTok{ length)}
\CommentTok{\#résultats intermédiaires}
\FunctionTok{print}\NormalTok{(res)}
\end{Highlighting}
\end{Shaded}

\begin{verbatim}
##         1         2         3         4 
## 0.5571831 0.4899055 0.5206802 0.4427351
\end{verbatim}

\begin{Shaded}
\begin{Highlighting}[]
\CommentTok{\#fonction de consolidation}
\NormalTok{moy}\OtherTok{\textless{}{-}}\FunctionTok{weighted.mean}\NormalTok{(res,poids)}
\FunctionTok{print}\NormalTok{(}\FunctionTok{paste}\NormalTok{(}\StringTok{\textquotesingle{}Moyenne parallel =\textquotesingle{}}\NormalTok{,moy))}
\end{Highlighting}
\end{Shaded}

\begin{verbatim}
## [1] "Moyenne parallel = 0.503159347549081"
\end{verbatim}

\begin{Shaded}
\begin{Highlighting}[]
\CommentTok{\#Eteindre les moteurs}
\NormalTok{parallel}\SpecialCharTok{::}\FunctionTok{stopCluster}\NormalTok{(clust)}
\CommentTok{\#affichage temps de calcul}
\FunctionTok{print}\NormalTok{(}\StringTok{\textquotesingle{}\textgreater{}\textgreater{} Temps de calcul total avec parSapply moy par bloc)\textquotesingle{}}\NormalTok{)}
\end{Highlighting}
\end{Shaded}

\begin{verbatim}
## [1] ">> Temps de calcul total avec parSapply moy par bloc)"
\end{verbatim}

\begin{Shaded}
\begin{Highlighting}[]
\CommentTok{\#temps de calcul}
\FunctionTok{toc}\NormalTok{()}
\end{Highlighting}
\end{Shaded}

\begin{verbatim}
## 4.48 sec elapsed
\end{verbatim}

\emph{Rapport sur l'usage des coeurs}

\begin{Shaded}
\begin{Highlighting}[]
\CommentTok{\# Rapport sur l\textquotesingle{}usage des coeurs}
\NormalTok{cl }\OtherTok{\textless{}{-}}\NormalTok{ snow}\SpecialCharTok{::}\FunctionTok{makeCluster}\NormalTok{(k) }
\NormalTok{ctime1 }\OtherTok{\textless{}{-}} \FunctionTok{snow.time}\NormalTok{(}\FunctionTok{clusterApply}\NormalTok{(cl,blocs,}\AttributeTok{fun=}\NormalTok{mean))}

\FunctionTok{plot}\NormalTok{(ctime1)}
\end{Highlighting}
\end{Shaded}

\includegraphics{calcul_parallel_files/figure-latex/unnamed-chunk-11-1.pdf}

\hypertarget{avec-les-packages-doparallel-et-foreach}{%
\section{Avec les packages Doparallel et
Foreach}\label{avec-les-packages-doparallel-et-foreach}}

\begin{Shaded}
\begin{Highlighting}[]
\CommentTok{\#nombre de cores à exploiter}
\CommentTok{\#k \textless{}{-} 4}
\FunctionTok{tic}\NormalTok{()}
\CommentTok{\#partition en blocs des donnees}
\NormalTok{blocs }\OtherTok{\textless{}{-}} \FunctionTok{split}\NormalTok{(a,}\DecValTok{1}\SpecialCharTok{+}\NormalTok{(}\DecValTok{1}\SpecialCharTok{:}\NormalTok{n)}\SpecialCharTok{\%\%}\NormalTok{k)}
\CommentTok{\#print(blocs)}

\CommentTok{\#configurer les cores}
\NormalTok{doParallel}\SpecialCharTok{::}\FunctionTok{registerDoParallel}\NormalTok{(k)}

\CommentTok{\#itérer sur les blocs}
\NormalTok{res }\OtherTok{\textless{}{-}}\NormalTok{ foreach}\SpecialCharTok{::}\FunctionTok{foreach}\NormalTok{(}\AttributeTok{b =}\NormalTok{ blocs, }\AttributeTok{.combine =}\NormalTok{ c) }\SpecialCharTok{\%dopar\%}\NormalTok{ \{}
  \FunctionTok{return}\NormalTok{(}\FunctionTok{mon\_min}\NormalTok{(b))}
\NormalTok{\}}

\CommentTok{\#résultats intermédiaires}
\CommentTok{\#print(res)}

\CommentTok{\#minimum global}
\FunctionTok{print}\NormalTok{(}\FunctionTok{paste}\NormalTok{(}\StringTok{\textquotesingle{}Min foreach/dopar =\textquotesingle{}}\NormalTok{,}\FunctionTok{mon\_min}\NormalTok{(res)))}
\end{Highlighting}
\end{Shaded}

\begin{verbatim}
## [1] "Min foreach/dopar = 0.126897253561765"
\end{verbatim}

\begin{Shaded}
\begin{Highlighting}[]
\CommentTok{\#stopper les cores}
\NormalTok{doParallel}\SpecialCharTok{::}\FunctionTok{stopImplicitCluster}\NormalTok{()}

\CommentTok{\#affichage temps de calcul}
\FunctionTok{print}\NormalTok{(}\StringTok{\textquotesingle{}\textgreater{}\textgreater{} Temps de calcul total avec foreach/dopar (split + min par bloc)\textquotesingle{}}\NormalTok{)}
\end{Highlighting}
\end{Shaded}

\begin{verbatim}
## [1] ">> Temps de calcul total avec foreach/dopar (split + min par bloc)"
\end{verbatim}

\begin{Shaded}
\begin{Highlighting}[]
\CommentTok{\#temps de calcul}
\FunctionTok{toc}\NormalTok{()}
\end{Highlighting}
\end{Shaded}

\begin{verbatim}
## 3.22 sec elapsed
\end{verbatim}

\hypertarget{la-regression-en-paralluxe8le}{%
\section{La regression en
parallèle}\label{la-regression-en-paralluxe8le}}

Données: Nous utilisons les données mtcars. Nous cherchons à expliquer
la consommation (mpg) en fonction des autres variables.

\begin{Shaded}
\begin{Highlighting}[]
\NormalTok{data}\OtherTok{\textless{}{-}}\NormalTok{(}\FunctionTok{data}\NormalTok{(mtcars))}
\CommentTok{\#View(mtcars)}
\end{Highlighting}
\end{Shaded}

\begin{Shaded}
\begin{Highlighting}[]
\NormalTok{alea }\OtherTok{\textless{}{-}} \FunctionTok{runif}\NormalTok{(}\FunctionTok{nrow}\NormalTok{(mtcars))}
\NormalTok{cle }\OtherTok{\textless{}{-}} \FunctionTok{ifelse}\NormalTok{(alea }\SpecialCharTok{\textless{}} \FloatTok{0.5}\NormalTok{, }\DecValTok{1}\NormalTok{, }\DecValTok{2}\NormalTok{)}
\NormalTok{blocs }\OtherTok{\textless{}{-}} \FunctionTok{split}\NormalTok{(mtcars,cle)}
\FunctionTok{print}\NormalTok{(blocs)}
\end{Highlighting}
\end{Shaded}

\begin{verbatim}
## $`1`
##                      mpg cyl  disp  hp drat    wt  qsec vs am gear carb
## Mazda RX4           21.0   6 160.0 110 3.90 2.620 16.46  0  1    4    4
## Datsun 710          22.8   4 108.0  93 3.85 2.320 18.61  1  1    4    1
## Hornet 4 Drive      21.4   6 258.0 110 3.08 3.215 19.44  1  0    3    1
## Hornet Sportabout   18.7   8 360.0 175 3.15 3.440 17.02  0  0    3    2
## Merc 240D           24.4   4 146.7  62 3.69 3.190 20.00  1  0    4    2
## Merc 230            22.8   4 140.8  95 3.92 3.150 22.90  1  0    4    2
## Merc 280            19.2   6 167.6 123 3.92 3.440 18.30  1  0    4    4
## Merc 280C           17.8   6 167.6 123 3.92 3.440 18.90  1  0    4    4
## Merc 450SE          16.4   8 275.8 180 3.07 4.070 17.40  0  0    3    3
## Cadillac Fleetwood  10.4   8 472.0 205 2.93 5.250 17.98  0  0    3    4
## Lincoln Continental 10.4   8 460.0 215 3.00 5.424 17.82  0  0    3    4
## Fiat 128            32.4   4  78.7  66 4.08 2.200 19.47  1  1    4    1
## Toyota Corona       21.5   4 120.1  97 3.70 2.465 20.01  1  0    3    1
## Dodge Challenger    15.5   8 318.0 150 2.76 3.520 16.87  0  0    3    2
## AMC Javelin         15.2   8 304.0 150 3.15 3.435 17.30  0  0    3    2
## Pontiac Firebird    19.2   8 400.0 175 3.08 3.845 17.05  0  0    3    2
## Fiat X1-9           27.3   4  79.0  66 4.08 1.935 18.90  1  1    4    1
## Ferrari Dino        19.7   6 145.0 175 3.62 2.770 15.50  0  1    5    6
## 
## $`2`
##                    mpg cyl  disp  hp drat    wt  qsec vs am gear carb
## Mazda RX4 Wag     21.0   6 160.0 110 3.90 2.875 17.02  0  1    4    4
## Valiant           18.1   6 225.0 105 2.76 3.460 20.22  1  0    3    1
## Duster 360        14.3   8 360.0 245 3.21 3.570 15.84  0  0    3    4
## Merc 450SL        17.3   8 275.8 180 3.07 3.730 17.60  0  0    3    3
## Merc 450SLC       15.2   8 275.8 180 3.07 3.780 18.00  0  0    3    3
## Chrysler Imperial 14.7   8 440.0 230 3.23 5.345 17.42  0  0    3    4
## Honda Civic       30.4   4  75.7  52 4.93 1.615 18.52  1  1    4    2
## Toyota Corolla    33.9   4  71.1  65 4.22 1.835 19.90  1  1    4    1
## Camaro Z28        13.3   8 350.0 245 3.73 3.840 15.41  0  0    3    4
## Porsche 914-2     26.0   4 120.3  91 4.43 2.140 16.70  0  1    5    2
## Lotus Europa      30.4   4  95.1 113 3.77 1.513 16.90  1  1    5    2
## Ford Pantera L    15.8   8 351.0 264 4.22 3.170 14.50  0  1    5    4
## Maserati Bora     15.0   8 301.0 335 3.54 3.570 14.60  0  1    5    8
## Volvo 142E        21.4   4 121.0 109 4.11 2.780 18.60  1  1    4    2
\end{verbatim}

\begin{Shaded}
\begin{Highlighting}[]
\CommentTok{\#reduce}
\NormalTok{reduce\_lm }\OtherTok{\textless{}{-}} \ControlFlowTok{function}\NormalTok{(D)\{}
 \CommentTok{\#nombre de lignes}
\NormalTok{ n }\OtherTok{\textless{}{-}} \FunctionTok{nrow}\NormalTok{(D)}
 \CommentTok{\#récupération de la cible}
\NormalTok{ y }\OtherTok{\textless{}{-}}\NormalTok{ D}\SpecialCharTok{$}\NormalTok{mpg}
 \CommentTok{\#prédictives}
\NormalTok{ X }\OtherTok{\textless{}{-}} \FunctionTok{as.matrix}\NormalTok{(D[,}\SpecialCharTok{{-}}\DecValTok{1}\NormalTok{])}
 \CommentTok{\#rajouter la constante en première colonne}
\NormalTok{ X }\OtherTok{\textless{}{-}} \FunctionTok{cbind}\NormalTok{(}\FunctionTok{rep}\NormalTok{(}\DecValTok{1}\NormalTok{,n),X)}
 \CommentTok{\#calcul de X\textquotesingle{}X}
\NormalTok{ XtX }\OtherTok{\textless{}{-}} \FunctionTok{t}\NormalTok{(X) }\SpecialCharTok{\%*\%}\NormalTok{ X}
 \CommentTok{\#calcul de X\textquotesingle{}y}
\NormalTok{ Xty }\OtherTok{\textless{}{-}} \FunctionTok{t}\NormalTok{(X) }\SpecialCharTok{\%*\%}\NormalTok{ y}
 \CommentTok{\#former une structure de liste}
\NormalTok{ res }\OtherTok{\textless{}{-}} \FunctionTok{list}\NormalTok{(}\AttributeTok{XtX =}\NormalTok{ XtX, }\AttributeTok{Xty =}\NormalTok{ Xty)}
 \CommentTok{\#renvoyer le tout}
 \FunctionTok{return}\NormalTok{(res)}
\NormalTok{\}}
\end{Highlighting}
\end{Shaded}

\begin{Shaded}
\begin{Highlighting}[]
\CommentTok{\#Demarrage des moteurs (workers)}
\NormalTok{clust }\OtherTok{\textless{}{-}}\NormalTok{ parallel}\SpecialCharTok{::}\FunctionTok{makeCluster}\NormalTok{(}\DecValTok{4}\NormalTok{)}
\CommentTok{\#lancement des min en parallele}
\NormalTok{res }\OtherTok{\textless{}{-}}\NormalTok{ parallel}\SpecialCharTok{::}\FunctionTok{parSapply}\NormalTok{(clust,blocs,}\AttributeTok{FUN =}\NormalTok{ reduce\_lm)}

\CommentTok{\#résultats intermédiaires}
\FunctionTok{print}\NormalTok{(res)}
\end{Highlighting}
\end{Shaded}

\begin{verbatim}
##     1           2          
## XtX numeric,121 numeric,121
## Xty numeric,11  numeric,11
\end{verbatim}

\begin{Shaded}
\begin{Highlighting}[]
\CommentTok{\#fonction de consolidation}
\CommentTok{\#consolidation}
\CommentTok{\#X\textquotesingle{}X}
\NormalTok{MXtX }\OtherTok{\textless{}{-}} \FunctionTok{matrix}\NormalTok{(}\DecValTok{0}\NormalTok{,}\AttributeTok{nrow=}\FunctionTok{ncol}\NormalTok{(mtcars),}\AttributeTok{ncol=}\FunctionTok{ncol}\NormalTok{(mtcars))}
\ControlFlowTok{for}\NormalTok{ (i }\ControlFlowTok{in} \FunctionTok{seq}\NormalTok{(}\DecValTok{1}\NormalTok{,}\FunctionTok{length}\NormalTok{(res)}\SpecialCharTok{{-}}\DecValTok{1}\NormalTok{,}\DecValTok{2}\NormalTok{))\{}
\NormalTok{ MXtX }\OtherTok{\textless{}{-}}\NormalTok{ MXtX }\SpecialCharTok{+}\NormalTok{ res[[i]]}
\NormalTok{\}}
\FunctionTok{print}\NormalTok{(MXtX)}
\end{Highlighting}
\end{Shaded}

\begin{verbatim}
##                     cyl       disp         hp       drat         wt       qsec
##        32.000   198.000    7383.10    4694.00   115.0900   102.9520    571.160
## cyl   198.000  1324.000   51872.40   32204.00   691.4000   679.4040   3475.560
## disp 7383.100 51872.400 2179627.47 1291364.40 25094.7960 27091.4888 128801.504
## hp   4694.000 32204.000 1291364.40  834278.00 16372.2800 16471.7440  81092.160
## drat  115.090   691.400   25094.80   16372.28   422.7907   358.7190   2056.914
## wt    102.952   679.404   27091.49   16471.74   358.7190   360.9011   1828.095
## qsec  571.160  3475.560  128801.50   81092.16  2056.9140  1828.0946  10293.480
## vs     14.000    64.000    1854.40    1279.00    54.0300    36.5580    270.670
## am     13.000    66.000    1865.90    1649.00    52.6500    31.3430    225.680
## gear  118.000   710.000   25650.30   17112.00   432.9500   366.5820   2097.460
## carb   90.000   604.000   23216.10   15776.00   321.2600   310.5020   1547.670
##            vs       am      gear      carb
##        14.000   13.000   118.000    90.000
## cyl    64.000   66.000   710.000   604.000
## disp 1854.400 1865.900 25650.300 23216.100
## hp   1279.000 1649.000 17112.000 15776.000
## drat   54.030   52.650   432.950   321.260
## wt     36.558   31.343   366.582   310.502
## qsec  270.670  225.680  2097.460  1547.670
## vs     14.000    7.000    54.000    25.000
## am      7.000   13.000    57.000    38.000
## gear   54.000   57.000   452.000   342.000
## carb   25.000   38.000   342.000   334.000
\end{verbatim}

\begin{Shaded}
\begin{Highlighting}[]
\CommentTok{\#X\textquotesingle{}y}
\NormalTok{MXty }\OtherTok{\textless{}{-}} \FunctionTok{matrix}\NormalTok{(}\DecValTok{0}\NormalTok{,}\AttributeTok{nrow=}\FunctionTok{ncol}\NormalTok{(mtcars),}\AttributeTok{ncol=}\DecValTok{1}\NormalTok{)}
\ControlFlowTok{for}\NormalTok{ (i }\ControlFlowTok{in} \FunctionTok{seq}\NormalTok{(}\DecValTok{2}\NormalTok{,}\FunctionTok{length}\NormalTok{(res),}\DecValTok{2}\NormalTok{))\{}
\NormalTok{ MXty }\OtherTok{\textless{}{-}}\NormalTok{ MXty }\SpecialCharTok{+}\NormalTok{ res[[i]]}
\NormalTok{\}}
\FunctionTok{print}\NormalTok{(MXty)}
\end{Highlighting}
\end{Shaded}

\begin{verbatim}
##            [,1]
##         642.900
## cyl    3693.600
## disp 128705.080
## hp    84362.700
## drat   2380.277
## wt     1909.753
## qsec  11614.745
## vs      343.800
## am      317.100
## gear   2436.900
## carb   1641.900
\end{verbatim}

\begin{Shaded}
\begin{Highlighting}[]
\CommentTok{\#Eteindre les moteurs}
\NormalTok{parallel}\SpecialCharTok{::}\FunctionTok{stopCluster}\NormalTok{(clust)}
\end{Highlighting}
\end{Shaded}

Estimation des paramètres de la régression. Les estimateurs â sont
produits à l'aide de procédure solve() de R.

\begin{Shaded}
\begin{Highlighting}[]
\CommentTok{\#coefficients de la régression}
\NormalTok{a.chapeau }\OtherTok{\textless{}{-}} \FunctionTok{solve}\NormalTok{(MXtX,MXty)}
\FunctionTok{print}\NormalTok{(a.chapeau)}
\end{Highlighting}
\end{Shaded}

\begin{verbatim}
##             [,1]
##      12.30337416
## cyl  -0.11144048
## disp  0.01333524
## hp   -0.02148212
## drat  0.78711097
## wt   -3.71530393
## qsec  0.82104075
## vs    0.31776281
## am    2.52022689
## gear  0.65541302
## carb -0.19941925
\end{verbatim}

Vérification - Procédure lm() de R. A titre de vérification, nous avons
effectué la régression à l'aide de la procédure lm() de R.

\begin{Shaded}
\begin{Highlighting}[]
\FunctionTok{print}\NormalTok{(}\FunctionTok{summary}\NormalTok{(}\FunctionTok{lm}\NormalTok{(mpg}\SpecialCharTok{\textasciitilde{}}\NormalTok{.,}\AttributeTok{data=}\NormalTok{mtcars)))}
\end{Highlighting}
\end{Shaded}

\begin{verbatim}
## 
## Call:
## lm(formula = mpg ~ ., data = mtcars)
## 
## Residuals:
##     Min      1Q  Median      3Q     Max 
## -3.4506 -1.6044 -0.1196  1.2193  4.6271 
## 
## Coefficients:
##             Estimate Std. Error t value Pr(>|t|)  
## (Intercept) 12.30337   18.71788   0.657   0.5181  
## cyl         -0.11144    1.04502  -0.107   0.9161  
## disp         0.01334    0.01786   0.747   0.4635  
## hp          -0.02148    0.02177  -0.987   0.3350  
## drat         0.78711    1.63537   0.481   0.6353  
## wt          -3.71530    1.89441  -1.961   0.0633 .
## qsec         0.82104    0.73084   1.123   0.2739  
## vs           0.31776    2.10451   0.151   0.8814  
## am           2.52023    2.05665   1.225   0.2340  
## gear         0.65541    1.49326   0.439   0.6652  
## carb        -0.19942    0.82875  -0.241   0.8122  
## ---
## Signif. codes:  0 '***' 0.001 '**' 0.01 '*' 0.05 '.' 0.1 ' ' 1
## 
## Residual standard error: 2.65 on 21 degrees of freedom
## Multiple R-squared:  0.869,  Adjusted R-squared:  0.8066 
## F-statistic: 13.93 on 10 and 21 DF,  p-value: 3.793e-07
\end{verbatim}

\hypertarget{map-reduce}{%
\section{MAP REDUCE}\label{map-reduce}}

\end{document}
